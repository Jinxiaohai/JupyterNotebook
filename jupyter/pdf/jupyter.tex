%#-*- coding:utf-8 -*-
\documentclass[11pt,UTF8,hyperref,openany]{ctexbook}
\usepackage{amsmath}             %%%%多种的公式环境和数学命令
\usepackage{amssymb}             %%%%数学符号生成命令
\usepackage{array}               %%%%数组和表格
\usepackage{booktabs}            %%%%水平的表格线
\usepackage{calc}                %%%%四则运算
\usepackage{caption}             %%%%插图和表格
% \usepackage{ctex}                %%%%中文字体
\usepackage{ctexcap}             %%%%中文字体和标题
\usepackage{color}
\usepackage{fancyhdr}            %%%%页眉页脚设置
\usepackage{graphicx}            %%%%插图
\usepackage{geometry}            %%%%版面尺寸控制
\geometry{left=2cm, right=2cm, top=2cm, bottom=2cm, head=2cm, foot=1cm}
% head=?cm, headmap=?cm
\usepackage{hyperref}            %%%%超链接
\usepackage{ifthen}              %%%%条件
\usepackage{longtable}           %%%%跨页表格
\usepackage{lineno}              %%%%行号控制
\usepackage{listings}            %%%%C++
\usepackage{multicol}            %%%%多栏
\usepackage{makeidx}             %%%%索引
\usepackage{ntheorem}            %%%%定理设置
\usepackage{paralist}            %%%%列表
\usepackage{tabularx}            %%%%表格的列宽
\usepackage{titlesec}            %%%%章节标题
\usepackage{fancyvrb}            %%%%抄录
\usepackage{fontspec}            %%%%字体
\usepackage{titletoc}            %%%%目录格式
\usepackage{xcolor}              %%%%颜色处理
\usepackage{xeCJK}               %%%%中日朝文字处理
%%%%%%%%%%%%%%%%%%%%%%%%%%%%%%%%%%%%%%%%%%%%%%%%%%%%%%%%%%
\author{xiaohai}
\date{1990/10/12}
\title{my title}

\begin{document}

%%%%%%%%%%%%%%%%%%%%%%%%%%%%%%%%%%%%%%%%%%%%%%%%%%%%%%%%%%
\renewcommand{\thefootnote}{}
\footnote{\heiti{ 作者简介}: 金小海,男}
\footnote{\heiti{ 现在单位}: Sinap}
% 将脚注的引用记数设为0.
\setcounter{footnote}{0}
\renewcommand{\thefootnote}{\arabic{footnote}}
\centering{\large{命令模式(按Esc开启)}}
\begin{itemize}
\item \textbf{Enter} : 转入编辑模式
\item {\textbf{Shift-Enter} : 运行本单元,选中下个单元}
\item {\textbf{Ctrl-Enter} : 运行本单元}
\item {\textbf{Alt-Enter} : 运行本单元,在其下插入新单元}
\item {\textbf{y} : 单元转入代码状态}
\item {\textbf{m} : 单元转入markdown状态}
\item {\textbf{r} : 单元转入raw状态}
\item {\textbf{1} : 设定一级标题}
\item {\textbf{2} : 设定二级标题}
\item {\textbf{3} : 设定三级标题}
\item {\textbf{4} : 设定四级标题}
\item {\textbf{5} : 设定五级标题}
\item {\textbf{6} : 设定六级标题}
\item {\textbf{k} : 选中上方单元}
\item {\textbf{j} : 选中下方单元}
\item {\textbf{Shift-k} : 扩大选中上方单元}
\item {\textbf{Shift-j} : 扩大选中下方单元}
\item {\textbf{a} : 在上方插入新单元}
\item {\textbf{b} : 在下方插入新单元}
\item {\textbf{x} : 剪切选中的单元}
\item {\textbf{c} : 复制选中的单元}
\item {\textbf{Shift-v} : 粘贴到上方单元}
\item {\textbf{v} : 粘贴到下方单元}
\item {\textbf{z} : 恢复删除的最后一个单元}
\item {\textbf{dd} : 删除选中的单元}
\item {\textbf{Shift-m} : 合并选中的单元}
\item {\textbf{Ctrl-s} : 文件存盘}
\item {\textbf{l} : 转换行号}
\item {\textbf{Shift-o} : 转换输出滚动}
\item {\textbf{h} : 显示快捷键帮助}
\item {\textbf{Shift-Space} : 向上滚动}
\item {\textbf{Space} : 向下滚动}
\end{itemize}

\centering{\large{编辑模式(按Enter键启动)}}
\begin{itemize}
\item {\textbf{tab} : 代码补全或缩进}
\item {\textbf{Shift-tab} : 提示}
\item {\textbf{Ctrl-]} : 缩进}
\item {\textbf{Ctrl-[} : 解除缩进}
\item {\textbf{Ctrl-a} : 全选}
\item {\textbf{Ctrl-z} : 复原}
\item {\textbf{Ctrl-y} : 再做}
\item {\textbf{Ctrl-home} : 跳到单元开头}
\item {\textbf{Ctrl-end} : 跳到单元结尾}
\item {\textbf{Esc} : 进入命令模式}
\item {\textbf{Shift-Enter} : 运行本单元,选中下个单元}
\item {\textbf{Ctrl-Enter} : 运行本单元}
\item {\textbf{Alt-Enter} : 运行本单元,在下面插入一个单元}
\item {\textbf{up} : 光标上移或转入上一个单元}
\item {\textbf{down} : 光标下移或转入下一个单元}
\end{itemize}

%%%%%%%%%%%%%%%%%%%%%%%%%%%%%%%%%%%%%%%%%%%%%%%%%%%%%%%%%%

\end{document}

%%% Local Variables:
%%% mode: latex
%%% TeX-master: "jupyter"
%%% End:
